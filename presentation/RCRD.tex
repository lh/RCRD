\documentclass{beamer}
\usetheme{Malmoe}
\usecolortheme{beaver}
\useoutertheme{miniframes}
\useinnertheme{circles}

% Required packages
\usepackage{verbatim}
\usepackage{xcolor}
\usepackage{graphicx}
\usepackage{listings}
\usepackage{textcomp}

% Include custom theme and color definitions
\usepackage{RCRDtheme}

% Configure spacing and presentation settings
\setlength{\parskip}{0.5em}
\setbeamertemplate{navigation symbols}{}  % Remove navigation symbols
\setbeamertemplate{footline}[frame number]  % Add frame numbers
\setbeamertemplate{frametitle continuation}[from second]

% Configure listings
\lstset{
  basicstyle=\ttfamily\small,
  columns=flexible,
  breaklines=true,
  frame=none,
  backgroundcolor=\color{white},
  showstringspaces=false,
  literate={├}{{\smash{\raisebox{-1ex}{\textcolor{black}{├}}}}}1
           {│}{{\textcolor{black}{│}}}1
           {└}{{\smash{\raisebox{-1ex}{\textcolor{black}{└}}}}}1
}

% Title page information
% Presentation structure
\AtBeginSection[]
{
  \begin{frame}
    \frametitle{Contents}
    \tableofcontents[currentsection]
  \end{frame}
}

\title{Retinal Detachment Risk Calculator\\Application Development}
\subtitle{A Test-Driven Development Approach}
\author{Luke Herbert}
\date{\today}

\begin{document}

\frame{\titlepage}

\begin{frame}
    \frametitle{Presentation Overview}
    \tableofcontents
\end{frame}

\section{Introduction}
\begin{frame}
    \frametitle{Project Overview}
    \begin{itemize}
        \item I wanted to make a little application to help me with my retinal detachment risk assessment. It was a side quest\dots \pause
        
        \item I had available a paper with various risk factors for retinal detachment. \pause
        
        \item In my head was a simple little app I could use on my phone or computer, put in parameters, draw a retinal detachment and get a risk output. \pause
    \end{itemize}
\end{frame}

\section{Project Requirements}
\begin{frame}
    \frametitle{Initial Constraints}
    \begin{alertblock}{Core Requirements}
        \begin{itemize}
            \item Had to be obvious to use
            \item No data sent off device
            \item Appear instantaneous to the user
            \item Graphic interface
        \end{itemize}
    \end{alertblock}
\end{frame}

\begin{frame}
    \frametitle{Constraint implications}
    \begin{tabular}{|l|l|}
        \hline
        Had to be obvious to use & \pause Down to my design \pause \\
        \hline
        No data sent off device & \pause Javascript? WASM? \pause \\
        \hline
        Appear instantaneous to the user & \pause Calculation on device \pause \\
        \hline
        Graphic interface & \pause Typescript and React? \\
        \hline
    \end{tabular}

    \vspace{1cm}

    Because I had been doing more object oriented stuff I chose TypeScript which is a superset of Javascript with types. React is a library for building user interfaces in TypeScript and seemed to be the most popular framework for building web applications, so I chose that. I didn't know either TypeScript or React before and I wanted this to be quick so I decided to use LLM to help me.
\end{frame}

\section{Development Process}
\begin{frame}
    \frametitle{LLM-Assisted Development Process}
    \begin{itemize}
        \item Work out the logic from the paper
        \item Design the calculation part
        \item Design the graphical part
    \end{itemize}
\end{frame}

\begin{frame}
    \frametitle{Logic from the paper}
    \begin{itemize}
        \item Supply the paper and ask LLM to extract the parameters
        \item Confirm the output manually
        \item Extract the regression equation from the paper
        \item Use the equation to calculate the risk and check against examples in the paper
    \end{itemize}

    \vspace{0.5cm}
    This was pretty straightforward and I got the logic from the paper.
\end{frame}

\begin{frame}
    \frametitle{Design the calculation part}
    \begin{itemize}
        \item I checked the maths separately
        \item Asked the LLM to create code to reproduce the calculation
        \item Checked against hand calculations and examples in the paper
    \end{itemize}

    \vspace{0.5cm}
    This was straightforward.
\end{frame}

\begin{frame}
    \frametitle{Design the graphical part}
    \begin{itemize}
        \item Initially asked the LLM to create code to produce a graphic of a clock face
        \item Added highlighting and the ability to select segments
        \item Added the logic around the selection of tears and detachment
    \end{itemize}
    
    \vspace{0.5cm}
    This was far from straightforward.
\end{frame}

\begin{frame}
    \frametitle{(Re-)Design the graphical part}
    \begin{itemize}
        \item Added highlighting and the ability to select segments
        \item Added the logic around the selection of tears and detachment
    \end{itemize}
    
    \vspace{0.5cm}
    This was far from straightforward!
\end{frame}

\section{Challenges and Solutions}
\begin{frame}
    \frametitle{LLM Integration Challenges}
    \begin{alertblock}{Key Challenges}
        \begin{itemize}
            \item If something wasn't quite right there was a tendency to go round in circles
            \item Hard to know when to stop
            \item Need for structured approach to avoid endless iterations
        \end{itemize}
    \end{alertblock}
\end{frame}

\begin{frame}
    \frametitle{Eventual LLM working pattern}
    \begin{block}{Documentation Strategy}
        \begin{itemize}
            \item Have a "meta" folder to contain:
            \begin{itemize}
                \item Design documents
                \item Examples
                \item Documents generated by the LLM
                \item A "next steps" document
            \end{itemize}
        \end{itemize}
    \end{block}
    
    \begin{block}{Testing Approach}
        \begin{itemize}
            \item Set up a testing framework:
            \begin{itemize}
                \item Make minimal tests to start
                \item Have the tests run after every change
                \item Use the tests to drive the development
            \end{itemize}
        \end{itemize}
    \end{block}
\end{frame}

\begin{frame}
    \frametitle{Using tests to drive development}
    As an example, the way things displayed had to be different for mobile and desktop. Working in parallel was too complicated so I got it working in one format and then wrote tests to check that the other format worked. These obviously failed, but I could feed back the results of the failing tests to the LLM until the tests passed.
\end{frame}

\section{Testing Framework}
\begin{frame}
    \frametitle{Testing Strategy Overview}
    \begin{enumerate}
        \item Test Organization
        \begin{itemize}
            \item Tests organized by view (mobile/desktop) and responsibility
            \item Clear separation between component, utility, and hook tests
            \item Tests grouped using describe blocks with clear descriptions
            \item Proper test isolation and setup/teardown
        \end{itemize}
    \end{enumerate}
\end{frame}

\begin{frame}
    \frametitle{Test-Driven Development cycle}
    \begin{alertblock}{Key TDD Steps}
        \begin{enumerate}
            \item Write/update test for a single change
            \item Verify test fails (red)
            \item Make minimal code changes
            \item Verify test passes (green)
            \item Refactor if needed
            \item Get approval before next change
        \end{enumerate}
    \end{alertblock}
\end{frame}

\begin{frame}
    \frametitle{Component Testing Structure Example}
    \begin{exampleblock}{Test Organization}
        \begin{itemize}
            \item RetinalCalculator tests split into:
            \begin{itemize}
                \item Container tests (view switching, layout)
                \item Mobile view tests (touch interactions)
                \item Desktop view tests (mouse interactions)
                \item Shared functionality tests
            \end{itemize}
        \end{itemize}
    \end{exampleblock}
\end{frame}

\begin{frame}  
    \begin{block}{Testing Principles Phase 1}
        I ran this when I first implemented my testing framework.
        
        There was an existing codebase that I wanted to document with tests before changing anything. 
        
        \begin{itemize}
            \item Never modify source code to make tests pass
            \item Document needed changes in meta/*.md files
            \item Skip failing tests with detailed comments
            \item Wait for explicit approval before implementation changes
            \item Maintain test isolation and clear boundaries
        \end{itemize}
    \end{block}
    \vspace{0.5cm}
    I would run a session on an aspect with the instructions for the LLM as above. When I had read and understood the needed changes, I would switch to Phase 2. \pause
\end{frame}

\begin{frame}
    \begin{block}{Testing Principles Phase 2}
    \begin{itemize}
        \item Never modify tests to make tests pass
        \item Make the changes described in meta/*.md files
        \item Reintroduce previously failing tests when addressed 
        \item Wait for explicit approval before implementation changes
        \item Maintain code isolation and clear boundaries
    \end{itemize}
\end{block}
\end{frame}

\begin{frame}[fragile]
    \frametitle{Testing Structure}
    \begin{block}{Directory Structure}
        Tests are organized following a clear directory structure that mirrors the source code:
    \end{block}
    
    \begin{lstlisting}[basicstyle=\ttfamily\small]
src/
|-- components/
|   |-- __tests__/          # Component tests
|   `-- ComponentName/
|       `-- __tests__/      # Specific component tests
`-- utils/
    `-- __tests__/          # Utility function tests
    \end{lstlisting}
\end{frame}

\begin{frame}[t]
    \frametitle{Testing Strategy}
    \begin{itemize}
        \item Comprehensive test cases \pause
        \item Edge case handling \pause
        \item Debug capabilities \pause
        \item Automated validation
    \end{itemize}
\end{frame}

\section{Results and Future Work}
\begin{frame}
    \frametitle{Key Development Achievements}
    \begin{exampleblock}{Technical Implementation}
        \begin{itemize}
            \item React-based implementation
            \item Modular component structure
            \item Utility functions for time conversion
            \item Robust testing framework
        \end{itemize}
    \end{exampleblock}
\end{frame}

\begin{frame}[t]
    \frametitle{Future Enhancements}
    \begin{block}{Planned Improvements}
    \begin{itemize}
        \item Additional time formats \pause
        \item Enhanced user interactions \pause
        \item Visual improvements \pause
        \item Extended test coverage
    \end{itemize}
    \end{block}
\end{frame}

\end{document}
