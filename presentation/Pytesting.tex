\documentclass{beamer}
\usetheme{Malmoe}
\usecolortheme{beaver}
\useoutertheme{miniframes}
\useinnertheme{circles}

% Required packages
\usepackage{verbatim}
\usepackage{xcolor}
\usepackage{graphicx}
\usepackage{listings}
\usepackage{textcomp}

% Include custom theme and color definitions
\usepackage{RCRDtheme}

% Configure spacing and presentation settings
\setlength{\parskip}{0.5em}
\setbeamertemplate{navigation symbols}{}  % Remove navigation symbols
\setbeamertemplate{footline}[frame number]  % Add frame numbers
\setbeamertemplate{frametitle continuation}[from second]

% Define custom language for configuration files
\lstdefinestyle{configstyle}{
    basicstyle=\ttfamily\small,
    columns=fullflexible,
    keepspaces=true,
    commentstyle=\color{green!60!black},
    keywordstyle=\color{red},
    stringstyle=\color{blue},
    frame=single
}

% Configure listings for Python code
\lstset{
    language=Python,
    basicstyle=\ttfamily\small,
    keywordstyle=\color{blue}\bfseries,
    stringstyle=\color{red},
    commentstyle=\color{green!60!black},
    numbers=left,
    numberstyle=\tiny,
    numbersep=5pt,
    frame=single,
    breaklines=true,
    breakatwhitespace=true,
    showstringspaces=false,
    columns=fullflexible,
    keepspaces=true
}

% Configure listings for bash
\lstdefinelanguage{bash}{
    keywords={},
    comment=[l]{\#},
    commentstyle=\color{green!60!black},
    basicstyle=\ttfamily\small,
    columns=fullflexible,
    keepspaces=true
}

% Configure listings for yaml
\lstdefinelanguage{yaml}{
    keywords={true,false,null,y,n},
    keywordstyle=\color{blue}\bfseries,
    basicstyle=\ttfamily\small,
    commentstyle=\color{green!60!black},
    morecomment=[l]{\#},
    stringstyle=\color{red},
    morestring=[b]',
    morestring=[b]",
    columns=fullflexible,
    keepspaces=true
}

% Title page information
% Presentation structure
\AtBeginSection[]
{
  \begin{frame}
    \frametitle{Contents}
    \tableofcontents[currentsection]
  \end{frame}
}

\title{Pytesting}
\subtitle{Running tests with pytest}
\author{Luke Herbert}
\date{\today}

\begin{document}

\frame{\titlepage}

\begin{frame}
    \frametitle{Presentation Overview}
    \tableofcontents
\end{frame}

\section{Setting Up pytest}
\begin{frame}[fragile]
    \frametitle{Initial Setup}
    \begin{block}{Install Required Packages}
        \begin{lstlisting}[language=bash]
pip install pytest==8.0.0
pip install pytest-cov==4.1.0
pip install hypothesis==6.98.0
pip install pytest-snapshot==0.9.0
        \end{lstlisting}
    \end{block}
    
    \begin{alertblock}{Project Configuration}
        Create \texttt{pytest.ini} at project root:
        \begin{lstlisting}[style=configstyle]
[pytest]
testpaths = tests
python_files = test_*.py
addopts = -v --cov=streamlit_app
        \end{lstlisting}
    \end{alertblock}
\end{frame}

\begin{frame}[fragile]
    \frametitle{Test Configuration}
    \begin{block}{conftest.py Setup}
        \begin{lstlisting}[language=Python]
import pytest
from streamlit_app import SmokeModel_MECC_Model

@pytest.fixture
def base_model():
    params = {
        'N_people': 50,
        'N_service': 1,
        'visit_prob': 0.1,
        'seed': 42
    }
    return SmokeModel_MECC_Model(**params)
        \end{lstlisting}
    \end{block}
\end{frame}

\section{Running Tests}
\begin{frame}[fragile]
    \frametitle{Basic Test Commands}
    \begin{exampleblock}{Common Usage}
        \begin{lstlisting}[language=bash]
# Run all tests with coverage
pytest --cov=streamlit_app

# Run specific test file
pytest tests/unit/test_parameters.py -v

# Run tests matching pattern
pytest -k "test_parameter" -v

# Generate HTML coverage report
pytest --cov=streamlit_app --cov-report=html
        \end{lstlisting}
    \end{exampleblock}
\end{frame}

\begin{frame}[fragile]
    \frametitle{Test Organization}
    \begin{lstlisting}[basicstyle=\ttfamily\small]
tests/
|-- conftest.py          # Shared fixtures
|-- unit/
|   |-- test_parameters.py
|   |-- test_smoke_model_initialization.py
|   |-- test_smoke_model_quit_mechanics.py
|   `-- test_smoke_model_relapse.py
`-- integration/
    |-- test_model_integration.py
    `-- test_service_scaling.py
    \end{lstlisting}
\end{frame}

\section{Writing Tests}
\begin{frame}[fragile]
    \frametitle{Test Structure Example}
    \begin{lstlisting}[language=Python]
def test_smoke_model_initialization(base_model):
    """Test model initialization with default params"""
    # Check initial state
    assert base_model.N_people == 50
    assert base_model.visit_prob == 0.1
    
    # Verify agent creation
    agents = [a for a in base_model.schedule.agents]
    assert len(agents) == 51  # 50 people + 1 service
    
    # Check random seed
    assert base_model.random.getstate() is not None
    \end{lstlisting}
\end{frame}

\begin{frame}[fragile]
    \frametitle{Testing Model Behavior}
    \begin{lstlisting}[language=Python]
def test_service_interaction(base_model):
    """Test interaction between people and services"""
    # Force visit probability to ensure contact
    base_model.visit_prob = 1.0
    
    # Run one step
    base_model.step()
    
    # Check data collection
    data = base_model.datacollector.get_model_vars_dataframe()
    assert "Total Contacts" in data.columns
    assert data["Total Contacts"].iloc[-1] > 0
    \end{lstlisting}
\end{frame}

\section{Continuous Integration}
\begin{frame}[fragile]
    \frametitle{Continuous Testing}
    \begin{block}{Using pytest-watch}
        \begin{lstlisting}[language=bash]
# Install pytest-watch
pip install pytest-watch

# Run with continuous monitoring
ptw --onpass "echo 'Success!'" --onfail "echo 'Failed!'"

# Watch specific tests
ptw tests/unit/ -- -v
        \end{lstlisting}
    \end{block}
\end{frame}

\begin{frame}[fragile]
    \frametitle{Pre-commit Hooks}
    \begin{alertblock}{Pre-commit Configuration}
        Create \texttt{.pre-commit-config.yaml}:
        \begin{lstlisting}[language=yaml]
repos:
- repo: local
  hooks:
  - id: pytest
    name: pytest
    entry: pytest
    language: system
    types: [python]
    pass_filenames: false
        \end{lstlisting}
    \end{alertblock}
\end{frame}

\section{Debugging Tests}
\begin{frame}[fragile]
    \frametitle{Debugging with PDB}
    \begin{block}{Using Python Debugger}
        \begin{lstlisting}[language=Python]
def test_complex_behavior(base_model):
    # Add breakpoint for debugging
    import pdb; pdb.set_trace()
    
    # Or use pytest.set_trace()
    import pytest; pytest.set_trace()
        \end{lstlisting}
    \end{block}
\end{frame}

\begin{frame}[fragile]
    \frametitle{Debug Command Options}
    \begin{alertblock}{Running Tests with Debug Mode}
        \begin{lstlisting}[language=bash]
# Print output during test run
pytest -s

# Show local variables on failure
pytest -l

# Enter PDB on first failure
pytest --pdb
        \end{lstlisting}
    \end{alertblock}
\end{frame}

\section{Coverage Analysis}
\begin{frame}[fragile]
    \frametitle{Coverage Reports}
    \begin{block}{Generating Coverage Data}
        \begin{lstlisting}[language=bash]
# Basic coverage
pytest --cov=streamlit_app

# Detailed terminal report
pytest --cov=streamlit_app --cov-report=term-missing

# HTML report
pytest --cov=streamlit_app --cov-report=html

# XML for CI systems
pytest --cov=streamlit_app --cov-report=xml
        \end{lstlisting}
    \end{block}
\end{frame}

\begin{frame}[fragile]
    \frametitle{Coverage Settings}
    \begin{alertblock}{Coverage Configuration}
        In \texttt{.coveragerc}:
        \begin{lstlisting}[style=configstyle]
[run]
source = streamlit_app
omit = tests/*
       setup.py
        \end{lstlisting}
    \end{alertblock}
\end{frame}

\end{document}
